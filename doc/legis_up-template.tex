\documentclass[12pt]{article}

\usepackage[backend=biber,style=authoryear]{biblatex}

\addbibresource{references.bib}

\usepackage{sbc-template}
\usepackage{graphicx,url}
\usepackage[utf8]{inputenc}
\usepackage[brazilian]{babel}
\usepackage{biblatex}

\sloppy

\title{LEGIS UP\\ Protótipo de Sistema de Automação para o Poder Legislativo Municipal}

\author{
	Adilson S. S. Moraes\inst{1}, 
	Derick S. Conceição\inst{1}, 
	Leon A. Q. Bentes\inst{1}, \\
	Márcio A. B. Moda\inst{1}, 
	Neuri M. F. Gato\inst{1}, 
	Yuri R. Martins\inst{1}
}


\address{Curso Superior em Análise e Desenvolvimento de Sistemas -- Instituto Federal do Pará (IFPA)\\
	CEP 68250-000 -- Óbidos -- PA -- Brasil
	\email{\{angel\}@ifpa.edu.br}
}

\begin{document}  

\maketitle

\begin{abstract}
  The present study aims to understand and develop a software prototype for automating and streamlining bureaucratic processes, including the management of legislative projects, agenda control, document management, transparency, and access to information, attendance and voting control, parliamentary committees, notifications, and alerts. This study aims to analyze and define the needs and characteristics of a software prototype project, providing a comprehensive overview of its functionalities and a technological solution aimed at automating legislative processes. Through this solution, it is believed that it will replace the physical archiving of documents, transitioning to a digital environment similar to the processes of the Chamber of Deputies and the Federal Senate. This proposal focuses on the development of an intuitive, comprehensive, and multi-platform software prototype, modernizing parliamentary activities and enhancing the efficiency and transparency of the process.
\end{abstract}
     
\begin{resumo} 
  O presente estudo busca entender e desenvolver um protótipo de software para automatização e agilização dos processos burocráticos dos parlamentos municipais, incluindo gerenciamento de projetos de lei, controle de pauta, gestão de documentos, transparência e acesso à informação, controle de presença e votação, comissões parlamentares, notificações e alertas. Este estudo tem como objetivo analisar e definir as necessidades e características de um projeto de protótipo de software, fornecendo uma visão abrangente das funcionalidades presentes e uma solução tecnológica visando a automatização nos processos legislativos Através dessa solução acredita-se que, substituirá o arquivamento físico de documentos, proporcionando uma transição para um ambiente digital similar aos processos da Câmara dos Deputados e do Senado Federal, sendo essa proposta no desenvolvimento de um protótipo de software intuitivo, abrangente e multiplataforma, modernizando as atividades parlamentares e melhorando a eficiência e transparência do processo.
\end{resumo}


\section{Introdução}

A palavra "parlamento" tem origem no grego antigo. Ela deriva do termo grego ``$\pi\alpha\rho\alpha\lambda\eta\mu\beta\epsilon\iota\nu$'' (paralēmpein), que significa ``falar ao lado de'' ou ``discutir''. O prefixo "para-" indica "ao lado de", enquanto "lēmpein" se refere a "falar". Essa raiz etimológica reflete a natureza essencial de um parlamento como um local onde os representantes do povo se reúnem para discutir e deliberar questões de interesse público O conceito de parlamento, como o entendemos hoje, tem suas origens na Grécia antiga, onde a democracia direta era praticada em cidades-estados como Atenas \parencite{sathler2016representaccao}. Os cidadãos se reuniam em assembleias para discutir e votar em questões políticas. Essa tradição de participação democrática influenciou o desenvolvimento das instituições parlamentares ao longo da história, inclusive na Europa medieval e moderna. Com o advento das revoluções industriais, a globalização tem causado desafios e impactos nos serviços públicos em todo o mundo, mais especificamente nas Câmaras Municipais \parencite{eirao2013sistema}.\\
 
O Poder Legislativo tem como função central a elaboração das leis, ao lado de exercer outras tarefas constitucionais como a apresentação pública de assuntos de interesse dos cidadãos, o debate sobre tais reivindicações de modo a agrega-las sob o interesse geral e a fiscalização política dos atos do executivo. Na lista de competências do Poder Legislativo Municipal, enumeradas pela Constituição, a principal é a de fazer, suspender, interpretar e revogar as leis de competência do Município. Outras funçes é de fiscalizar e controlar os atos do Poder Executivo; funções administrativas internas de organização de seus serviços e uma função política adicional: a de representar o povo em suas queixas e reivindicações, operando como uma ouvidoria geral da sociedade.\\

Chama-se de Processo Legislativo todo conjunto de atos realizados pelos órgãos do Poder Legislativo, de acordo com regras previamente fixadas, para elaborar normas jurídicas (emendas à Constituição, leis complementares, leis ordinárias e outros tipos normativos dispostos no art. 59 da Constituição Federal).  

\begin{quote}
``Processo legislativo é o conjunto de ações realizadas pelos órgãos do poder legislativo com o objetivo de proceder à elaboração das leis sejam elas constitucionais complementares e ordinárias bem como as resoluções e decretos legislativos''. (BRASIL, 2011).
\end{quote}

Nesse contexto, atualmente muitos Poderes Legislativos ainda não usufruem de um sistema informatizado de processo legislativo, seja para acompanhamento de pautas, tramitações de projetos de leis, acesso às leis, que ofereça aos cidadãos a transparência dos atos legislativos, assim como a serviços que requerem celeridade, redução de custos, organização e clareza de informações.\\

A automatização dos processos legislativos por meio de um protótipo de software representaria um avanço significativo na modernização e na eficiência do Poder Legislativo. Uma solução tecnológica que atenderia às necessidades específicas do contexto legislativo, proporcionando uma gestão mais eficiente e transparente das atividades legislativas.


\section{Objetivos}
\subsection{Objetivos Específicos}
\section{Fundamentação Teórica}
\subsection{Poder Legislativo Municipal}
\subsection{Modelagem e Análise de Sistemas}
\section{Coleta de Requisitos e Arquitetura de Software}
\section{Diagrama de Casos de Uso}
\section{Diagrama de Atividades}
\section{Diagrama de Classes}
\section{Modelagem de Banco da Dados}
\subsection{Protótipo do Legis Up}
\section{Resultados Esperados}
\section{Considerações Finais}


%\printbibliography

\end{document}